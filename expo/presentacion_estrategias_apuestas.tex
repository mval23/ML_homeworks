\documentclass[aspectratio=169]{beamer}
\usepackage[utf8]{inputenc}
\usepackage[spanish]{babel}
\usepackage{graphicx}
\usepackage{amsmath}
\usepackage{amssymb}
\usepackage{tikz}
\usepackage{pgfplots}
\usepackage{booktabs}
\usepackage{xcolor}

% Tema y colores
\usetheme{Madrid}
\usecolortheme{seahorse}

% Colores personalizados
\definecolor{azulprincipal}{RGB}{0,102,204}
\definecolor{rojoperdida}{RGB}{204,51,51}
\definecolor{verdeganancia}{RGB}{0,153,76}
\definecolor{naranjaatencion}{RGB}{255,153,0}

\title{Sistema Inteligente de Inversión en Apuestas Deportivas}
\subtitle{¿Es posible ganar dinero apostando de forma científica?}
\author{Análisis de Estrategias y Simulación de Resultados}
\date{Julio 2025}

\begin{document}

% Diapositiva de título
\begin{frame}
\titlepage
\end{frame}

% Agenda
\begin{frame}{¿De qué vamos a hablar hoy?}
\tableofcontents
\end{frame}

\section{La Pregunta del Millón: ¿Se puede ganar apostando?}

\begin{frame}{La gran pregunta que todos nos hacemos}
\begin{center}
\Large
\textcolor{azulprincipal}{\textbf{¿Es posible ganar dinero apostando de forma consistente?}}
\end{center}

\vspace{0.5cm}

\begin{columns}
\begin{column}{0.5\textwidth}
\textbf{Lo que dice la gente común:}
\begin{itemize}
\item "Las casas siempre ganan"
\item "Es pura suerte"
\item "Solo pierdes dinero"
\item "Es imposible predecir deportes"
\end{itemize}
\end{column}
\begin{column}{0.5\textwidth}
\textbf{Lo que dice la ciencia:}
\begin{itemize}
\item \textcolor{verdeganancia}{Existen patrones estadísticos}
\item \textcolor{azulprincipal}{Se pueden crear modelos predictivos}
\item \textcolor{naranjaatencion}{Requiere disciplina y método}
\item \textcolor{rojoperdida}{El riesgo siempre existe}
\end{itemize}
\end{column}
\end{columns}

\vspace{0.5cm}
\begin{center}
\textbf{Nuestro objetivo: Responder esta pregunta con datos y ciencia}
\end{center}
\end{frame}

\begin{frame}{¿Qué hicimos para responder esta pregunta?}
\begin{center}
\textbf{Creamos un \textcolor{azulprincipal}{laboratorio virtual} para probar diferentes formas de apostar}
\end{center}

\vspace{0.5cm}

\begin{tikzpicture}[scale=0.8, every node/.style={transform shape}]
% Proceso paso a paso
\node[draw, rectangle, fill=azulprincipal!30, minimum width=2.5cm, minimum height=1cm] at (0,0) {Datos\\Históricos};
\node[draw, rectangle, fill=verdeganancia!30, minimum width=2.5cm, minimum height=1cm] at (3.5,0) {Estrategias\\de Apuesta};
\node[draw, rectangle, fill=naranjaatencion!30, minimum width=2.5cm, minimum height=1cm] at (7,0) {Simulación\\Virtual};
\node[draw, rectangle, fill=rojoperdida!30, minimum width=2.5cm, minimum height=1cm] at (10.5,0) {Resultados\\y Análisis};

% Flechas
\draw[->] (1.25,0) -- (2.25,0);
\draw[->] (4.75,0) -- (5.75,0);
\draw[->] (8.25,0) -- (9.25,0);

% Explicaciones debajo
\node at (0,-1.5) {\small Miles de partidos};
\node at (3.5,-1.5) {\small 3 métodos diferentes};
\node at (7,-1.5) {\small Como un videojuego};
\node at (10.5,-1.5) {\small ¿Funcionó?};
\end{tikzpicture}

\vspace{0.5cm}
\textbf{Pensémoslo como un videojuego donde probamos diferentes estrategias para ver cuál funciona mejor}
\end{frame}

\section{¿Qué son las Estrategias de Apuestas?}

\begin{frame}{Explicación simple: ¿Qué es una estrategia de apuestas?}
\begin{block}{Una estrategia es simplemente un conjunto de reglas}
Una estrategia de apuestas es como una \textbf{receta de cocina} pero para decidir:
\begin{itemize}
\item ¿En qué partido apostar?
\item ¿A qué resultado apostar?
\item ¿Cuánto dinero apostar?
\item ¿Cuándo parar?
\end{itemize}
\end{block}

\vspace{0.5cm}

\textbf{Ejemplo de la vida real:}
\begin{itemize}
\item \textcolor{azulprincipal}{\textbf{Estrategia personal:}} "Solo apuesto cuando estoy muy seguro"
\item \textcolor{verdeganancia}{\textbf{Estrategia científica:}} "Apuesto cuando mis cálculos dicen que hay valor"
\end{itemize}

\vspace{0.3cm}
\begin{center}
\textbf{La diferencia: Una usa intuición, la otra usa matemáticas y datos}
\end{center}
\end{frame}

\begin{frame}{Las 3 estrategias que pusimos a prueba}
\begin{columns}
\begin{column}{0.33\textwidth}
\begin{block}{\textcolor{azulprincipal}{\textbf{1. El Favorito}}}
\textbf{Regla simple:}
"Siempre apuesta al equipo que tiene más probabilidades de ganar"

\vspace{0.3cm}
\textbf{Lógica:}
Si las casas ponen cuotas bajas, es porque el equipo tiene muchas posibilidades
\end{block}
\end{column}

\begin{column}{0.33\textwidth}
\begin{block}{\textcolor{verdeganancia}{\textbf{2. Value Betting}}}
\textbf{Regla inteligente:}
"Solo apuesta cuando encuentres una 'ganga'"

\vspace{0.3cm}
\textbf{Lógica:}
Como buscar ofertas en el supermercado, pero con cuotas deportivas
\end{block}
\end{column}

\begin{column}{0.33\textwidth}
\begin{block}{\textcolor{rojoperdida}{\textbf{3. Aleatorio}}}
\textbf{Regla de control:}
"Apuesta al azar para comparar"

\vspace{0.3cm}
\textbf{Lógica:}
Para saber si las otras estrategias realmente funcionan
\end{block}
\end{column}
\end{columns}

\vspace{0.5cm}
\begin{center}
\textbf{Probamos cada estrategia con miles de partidos reales para ver cuál funciona mejor}
\end{center}
\end{frame}

\begin{frame}{¿Qué es "Value Betting"? Explicación con ejemplo cotidiano}
\begin{center}
\Large \textbf{Value Betting = Encontrar "ofertas" en las apuestas}
\end{center}

\vspace{0.5cm}

\begin{columns}
\begin{column}{0.5\textwidth}
\textbf{Ejemplo del supermercado:}
\begin{itemize}
\item Producto normal: 10€
\item Precio de oferta: 6€
\item \textcolor{verdeganancia}{\textbf{¡Hay valor!}} Ahorras 4€
\end{itemize}

\vspace{0.3cm}
\textbf{En apuestas deportivas:}
\begin{itemize}
\item Tu cálculo: 70\% de ganar
\item Cuota de la casa: Implica 50\% de ganar
\item \textcolor{verdeganancia}{\textbf{¡Hay valor!}} La casa subestima
\end{itemize}
\end{column}

\begin{column}{0.5\textwidth}
\begin{tikzpicture}[scale=0.9]
% Gráfico de valor
\draw[fill=rojoperdida!30] (0,0) rectangle (2,2);
\node at (1,1) {\small Cuota\\Casa};
\node at (1,2.3) {50\%};

\draw[fill=verdeganancia!30] (2.5,0) rectangle (4.5,3);
\node at (3.5,1.5) {\small Tu\\Cálculo};
\node at (3.5,3.3) {70\%};

\draw[<->] (2.1,2.5) -- (2.4,2.5);
\node at (2.25,2.8) {\tiny Valor};

\node at (2.25,-0.5) {\textbf{¡Apuesta!}};
\end{tikzpicture}
\end{column}
\end{columns}

\vspace{0.5cm}
\begin{center}
\textbf{El truco está en ser mejor que la casa calculando probabilidades}
\end{center}
\end{frame}

\section{Gestión del Dinero: La Clave del Éxito}

\begin{frame}{¿Por qué la gestión del dinero es lo más importante?}
\begin{alertblock}{La verdad que nadie te cuenta}
\Large
\textbf{No importa qué tan bueno seas prediciendo} \\
Si no sabes gestionar tu dinero, \textcolor{rojoperdida}{\textbf{perderás todo}}
\end{alertblock}

\vspace{0.5cm}

\begin{columns}
\begin{column}{0.5\textwidth}
\textbf{\textcolor{rojoperdida}{Ejemplo de lo que NO hacer:}}
\begin{itemize}
\item Tienes 1,000€
\item Apuestas 500€ en un partido
\item Pierdes: Te quedan 500€
\item Apuestas 250€ en el siguiente
\item Pierdes: Te quedan 250€
\item \textbf{¡En 2 apuestas perdiste 75\%!}
\end{itemize}
\end{column}

\begin{column}{0.5\textwidth}
\textbf{\textcolor{verdeganancia}{Ejemplo de gestión inteligente:}}
\begin{itemize}
\item Tienes 1,000€
\item Apuestas 20€ (2\%) en cada partido
\item Puedes perder 50 veces seguidas
\item Tiempo para recuperarte
\item \textbf{¡Sostenibilidad a largo plazo!}
\end{itemize}
\end{column}
\end{columns}

\vspace{0.5cm}
\begin{center}
\textbf{Regla de oro: Nunca apostar más del 2-3\% de tu dinero total}
\end{center}
\end{frame}

\begin{frame}{¿Qué es el "drawdown"? El concepto más importante}
\begin{center}
\Large
\textbf{Drawdown = ¿Cuánto dinero puedes llegar a perder?}
\end{center}

\vspace{0.5cm}

\begin{tikzpicture}[scale=1.1]
% Gráfico de evolución del capital
\draw[->] (0,0) -- (10,0) node[right] {Tiempo};
\draw[->] (0,0) -- (0,4) node[above] {Dinero (€)};

% Línea de capital
\draw[thick, azulprincipal] (0,2) -- (2,3) -- (4,2.5) -- (5,1.5) -- (7,2.8) -- (9,3.5);

% Mostrar drawdown
\draw[dashed, rojoperdida] (2,3) -- (5,3);
\draw[dashed, rojoperdida] (5,1.5) -- (5,3);
\draw[<->] (5.2,1.5) -- (5.2,3);
\node[right] at (5.2,2.25) {\textcolor{rojoperdida}{\textbf{Drawdown}}};

% Etiquetas
\node at (2,3.3) {\small Pico};
\node at (5,1.2) {\small Valle};

% Líneas de referencia
\draw[dashed] (0,2) -- (10,2);
\node[left] at (0,2) {1000€};
\node[left] at (0,3) {1500€};
\node[left] at (0,1) {500€};
\end{tikzpicture}

\vspace{0.3cm}
\textbf{En este ejemplo:}
\begin{itemize}
\item Máximo que tuviste: 1,500€
\item Mínimo después del máximo: 750€
\item \textcolor{rojoperdida}{\textbf{Drawdown: 50\%}} (perdiste la mitad desde el pico)
\end{itemize}

\begin{center}
\textbf{Prepararse mentalmente para drawdowns es esencial para el éxito}
\end{center}
\end{frame}

\section{Los Resultados: ¿Qué Descubrimos?}

\begin{frame}{Los números no mienten: Esto es lo que encontramos}
\begin{center}
\Large
\textbf{Resultados de probar las estrategias con miles de partidos reales}
\end{center}

\vspace{0.5cm}

\begin{center}
\begin{tabular}{lccc}
\toprule
\textbf{Estrategia} & \textbf{ROI} & \textbf{Éxito} & \textbf{Riesgo} \\
\midrule
\textcolor{azulprincipal}{\textbf{Favorito}} & +2.5\% & 58\% & Medio \\
\textcolor{verdeganancia}{\textbf{Value Betting}} & -1.2\% & 45\% & Alto \\
\textcolor{rojoperdida}{\textbf{Aleatorio}} & -5.8\% & 33\% & Muy Alto \\
\bottomrule
\end{tabular}
\end{center}

\vspace{0.5cm}

\textbf{¿Qué significa esto en español?}
\begin{itemize}
\item \textcolor{azulprincipal}{\textbf{Apostar al favorito:}} Por cada 100€, ganas 2.50€
\item \textcolor{verdeganancia}{\textbf{Value betting:}} Perdiste 1.20€ por cada 100€
\item \textcolor{rojoperdida}{\textbf{Apostar al azar:}} Perdiste 5.80€ por cada 100€
\end{itemize}

\vspace{0.3cm}
\begin{center}
\textbf{Conclusión: Solo la estrategia del favorito fue rentable}
\end{center}
\end{frame}

\begin{frame}{¿Por qué la estrategia del favorito funcionó mejor?}
\begin{columns}
\begin{column}{0.6\textwidth}
\textbf{Razones del éxito:}

\begin{enumerate}
\item \textcolor{verdeganancia}{\textbf{Simplicidad:}} Fácil de seguir sin errores
\item \textcolor{azulprincipal}{\textbf{Información del mercado:}} Las casas son buenas prediciendo favoritos
\item \textcolor{naranjaatencion}{\textbf{Menor volatilidad:}} Menos sorpresas desagradables
\item \textcolor{verdeganancia}{\textbf{Psicología:}} Más fácil mantener la disciplina
\end{enumerate}

\vspace{0.5cm}
\textbf{¿Por qué value betting falló?}
\begin{itemize}
\item Muy difícil ser mejor que las casas
\item Mayor riesgo por apuesta
\item Necesita más capital y paciencia
\end{itemize}
\end{column}

\begin{column}{0.4\textwidth}
\begin{tikzpicture}[scale=0.8]
% Gráfico de barras simple
\draw[fill=azulprincipal!70] (0,0) rectangle (1,2.5);
\node at (0.5,-0.3) {\small Favorito};
\node at (0.5,2.8) {\textbf{+2.5\%}};

\draw[fill=naranjaatencion!70] (1.5,0) rectangle (2.5,1);
\node at (2,-0.3) {\small Value};
\node at (2,1.3) {-1.2\%};

\draw[fill=rojoperdida!70] (3,0) rectangle (4,0.3);
\node at (3.5,-0.3) {\small Random};
\node at (3.5,0.6) {-5.8\%};

% Línea de referencia
\draw[dashed] (0,2) -- (4.5,2);
\node at (4.7,2) {\small 0\%};

\node at (2,-1) {\textbf{ROI por Estrategia}};
\end{tikzpicture}
\end{column}
\end{columns}

\vspace{0.5cm}
\begin{center}
\textbf{Lección: A veces lo simple es lo que mejor funciona}
\end{center}
\end{frame}

\begin{frame}{La importancia del tamaño de apuesta: Análisis de sensibilidad}
\begin{center}
\textbf{¿Qué pasa si apostamos más o menos dinero en cada partido?}
\end{center}

\vspace{0.5cm}

\begin{tikzpicture}[scale=1.1]
% Gráfico de línea mostrando ROI vs tamaño de apuesta
\draw[->] (0,0) -- (8,0) node[right] {Tamaño de apuesta (\% del capital)};
\draw[->] (0,0) -- (0,4) node[above] {ROI (\%)};

% Curva de ROI
\draw[thick, azulprincipal] (0.5,1) -- (1.5,2.8) -- (2.5,3.2) -- (4,2.1) -- (6,0.5) -- (7,-1);

% Puntos importantes
\fill[verdeganancia] (2.5,3.2) circle (0.1);
\node[above] at (2.5,3.2) {\textbf{Óptimo}};
\node[below] at (2.5,2.8) {3\%};

% Zonas
\draw[dashed, rojoperdida] (5,0) -- (5,4);
\node at (6,3.5) {\textcolor{rojoperdida}{\textbf{Zona de riesgo}}};
\node at (2,0.5) {\textcolor{verdeganancia}{\textbf{Zona segura}}};

% Etiquetas del eje
\node at (1,-0.3) {1\%};
\node at (2.5,-0.3) {3\%};
\node at (4,-0.3) {5\%};
\node at (6,-0.3) {8\%};
\node at (7,-0.3) {10\%};

\node at (-0.3,1) {1\%};
\node at (-0.3,2) {2\%};
\node at (-0.3,3) {3\%};
\end{tikzpicture}

\vspace{0.5cm}
\textbf{Descubrimiento importante:}
\begin{itemize}
\item \textcolor{verdeganancia}{\textbf{2-3\% por apuesta:}} Máximo rendimiento
\item \textcolor{naranjaatencion}{\textbf{1\%:}} Muy seguro pero poco rentable
\item \textcolor{rojoperdida}{\textbf{>5\%:}} Muy arriesgado, puedes perder todo
\end{itemize}
\end{frame}

\section{¿Es Realista Ganar Dinero Apostando?}

\begin{frame}{La verdad completa: Ventajas y desventajas}
\begin{columns}
\begin{column}{0.5\textwidth}
\textbf{\textcolor{verdeganancia}{Lo que SÍ es posible:}}
\begin{itemize}
\item Ganar pequeños retornos (2-5\% anual)
\item Con mucha disciplina y paciencia
\item Usando estrategias simples y probadas
\item Con capital suficiente para soportar pérdidas
\item Como complemento, no como trabajo principal
\end{itemize}
\end{column}

\begin{column}{0.5\textwidth}
\textbf{\textcolor{rojoperdida}{Lo que NO es realista:}}
\begin{itemize}
\item Hacerse rico rápidamente
\item Ganar siempre
\item No perder dinero nunca
\item Vivir solo de las apuestas
\item Funcionar sin conocimientos y disciplina
\end{itemize}
\end{column}
\end{columns}

\vspace{0.5cm}
\begin{alertblock}{Comparación con otros tipos de inversión}
\begin{center}
\begin{tabular}{lcc}
\textbf{Tipo de inversión} & \textbf{Retorno anual} & \textbf{Riesgo} \\
\midrule
Cuenta de ahorros & 1-2\% & Muy bajo \\
Fondos de inversión & 4-7\% & Medio \\
\textcolor{azulprincipal}{\textbf{Apuestas deportivas}} & \textcolor{azulprincipal}{\textbf{2-5\%}} & \textcolor{azulprincipal}{\textbf{Alto}} \\
\end{tabular}
\end{center}
\end{alertblock}
\end{frame}

\begin{frame}{Los requisitos para tener éxito (si decides intentarlo)}
\begin{block}{Capital mínimo necesario}
\textbf{Para intentar esto seriamente necesitas:}
\begin{itemize}
\item \textcolor{azulprincipal}{\textbf{Mínimo 10,000-50,000€}} de capital inicial
\item Dinero que \textcolor{rojoperdida}{\textbf{puedas permitirte perder}} completamente
\item Capacidad de no tocar el dinero por \textbf{6-12 meses}
\end{itemize}
\end{block}

\begin{block}{Habilidades y disciplina requeridas}
\textbf{Aspectos personales imprescindibles:}
\begin{itemize}
\item \textcolor{verdeganancia}{\textbf{Disciplina férrea}} para seguir las reglas siempre
\item \textcolor{azulprincipal}{\textbf{Paciencia}} para soportar rachas de pérdidas
\item \textcolor{naranjaatencion}{\textbf{Conocimientos}} básicos de estadística y finanzas
\item \textcolor{rojoperdida}{\textbf{Control emocional}} para no apostar por impulso
\end{itemize}
\end{block}

\vspace{0.3cm}
\begin{center}
\textbf{Si no cumples TODOS estos requisitos, mejor invierte en fondos tradicionales}
\end{center}
\end{frame}

\begin{frame}{Los riesgos que debes conocer antes de empezar}
\begin{alertblock}{Riesgos financieros}
\begin{itemize}
\item \textcolor{rojoperdida}{\textbf{Pérdida total del capital}} es posible y probable
\item \textbf{Drawdowns del 20-30\%} son normales
\item Los \textbf{retornos pasados no garantizan} retornos futuros
\item Las estrategias pueden \textbf{dejar de funcionar} con el tiempo
\end{itemize}
\end{alertblock}

\begin{alertblock}{Riesgos personales y legales}
\begin{itemize}
\item \textcolor{naranjaatencion}{\textbf{Adicción al juego}} es un riesgo real
\item \textbf{Estrés emocional} por las pérdidas
\item \textbf{Problemas legales y fiscales} si no declaras ganancias
\item \textbf{Limitaciones de cuentas} por parte de las casas de apuestas
\end{itemize}
\end{alertblock}

\vspace{0.5cm}
\begin{center}
\textbf{Reflexiona seriamente si los riesgos merecen los retornos potenciales}
\end{center}
\end{frame}

\section{Lecciones Aprendidas y Conclusiones}

\begin{frame}{Las 5 lecciones más importantes del proyecto}
\begin{enumerate}
\item \textbf{\textcolor{verdeganancia}{La gestión del dinero es más importante que predecir bien}}
   \begin{itemize}
   \item Apostar solo 2-3\% del capital por partido
   \item Prepararse para drawdowns del 20-30\%
   \end{itemize}

\item \textbf{\textcolor{azulprincipal}{Las estrategias simples funcionan mejor}}
   \begin{itemize}
   \item "Apostar al favorito" superó a métodos más complejos
   \item La simplicidad facilita la disciplina
   \end{itemize}

\item \textbf{\textcolor{naranjaatencion}{Los retornos realistas son modestos}}
   \begin{itemize}
   \item 2-5\% anual es un objetivo realista
   \item Olvidate de hacerte rico rápido
   \end{itemize}

\item \textbf{\textcolor{rojoperdida}{El riesgo de pérdida total es real}}
   \begin{itemize}
   \item Solo usar dinero que puedas permitirte perder
   \item Tener planes de contingencia
   \end{itemize}

\item \textbf{\textcolor{azulprincipal}{La disciplina lo es todo}}
   \begin{itemize}
   \item Seguir las reglas incluso cuando pierdes
   \item No apostar por emociones
   \end{itemize}
\end{enumerate}
\end{frame}

\begin{frame}{¿Vale la pena intentarlo? Nuestra recomendación}
\begin{columns}
\begin{column}{0.5\textwidth}
\textbf{\textcolor{verdeganancia}{SÍ vale la pena si:}}
\begin{itemize}
\item Tienes capital suficiente (+10,000€)
\item Entiendes y aceptas los riesgos
\item Puedes mantener disciplina férrea
\item Lo ves como un experimento/hobby
\item Tienes otras fuentes de ingresos
\item Te interesa aprender sobre finanzas
\end{itemize}
\end{column}

\begin{column}{0.5\textwidth}
\textbf{\textcolor{rojoperdida}{NO vale la pena si:}}
\begin{itemize}
\item Necesitas el dinero para vivir
\item Buscas ganancias rápidas
\item No tienes disciplina emocional
\item Ya tienes problemas con el juego
\item No entiendes los conceptos básicos
\item Esperas retornos altos garantizados
\end{itemize}
\end{column}
\end{columns}

\vspace{0.5cm}
\begin{alertblock}{Nuestra recomendación final}
\textbf{Para el 95\% de las personas: Invierte en fondos indexados tradicionales}
\begin{itemize}
\item Menores riesgos, retornos similares o mejores
\item Sin estrés emocional ni riesgo de adicción
\item Regulación y protección legal
\end{itemize}
\end{alertblock}
\end{frame}

\begin{frame}{¿Qué aporta este estudio al mundo?}
\begin{block}{Contribuciones científicas}
\textbf{Este proyecto es importante porque:}
\begin{itemize}
\item \textcolor{azulprincipal}{\textbf{Primera metodología científica}} completa para evaluar estrategias de apuestas
\item \textcolor{verdeganancia}{\textbf{Código abierto}} para que otros puedan replicar y mejorar
\item \textcolor{naranjaatencion}{\textbf{Estándares de transparencia}} para una industria tradicionalmente opaca
\end{itemize}
\end{block}

\begin{block}{Aplicaciones futuras}
\textbf{Esta metodología se puede usar para:}
\begin{itemize}
\item Evaluar estrategias en otros deportes
\item Análisis de mercados financieros alternativos
\item Investigación académica en sports analytics
\item Desarrollo de herramientas de educación financiera
\end{itemize}
\end{block}

\vspace{0.5cm}
\begin{center}
\textbf{El objetivo final: Que las personas tomen decisiones informadas basadas en datos reales}
\end{center}
\end{frame}

\begin{frame}
\begin{center}
\Huge \textcolor{azulprincipal}{\textbf{¿Preguntas?}}

\vspace{1cm}
\Large 
\textbf{Gracias por su atención}

\vspace{0.8cm}
\large
\textbf{Sistema Inteligente de Inversión en Apuestas Deportivas}\\
\textit{Análisis científico de estrategias y gestión de riesgo}

\vspace{0.5cm}
\normalsize
Recuerden: \textcolor{rojoperdida}{\textbf{Solo apostar dinero que puedan permitirse perder}}

\vspace{0.3cm}
Julio 2025
\end{center}
\end{frame}

\end{document}
